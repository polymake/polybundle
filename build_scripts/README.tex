\documentclass[a4paper]{amsart}
\usepackage{color}

\definecolor{urlcol}{cmyk}{.1,.7,1,0}
\usepackage[colorlinks,urlcolor=urlcol]{hyperref}
\usepackage{xspace}

\newcommand{\macversion}{MACVERSION}
\newcommand{\polymakeversion}{Release 3.2 of January 29, 2018\xspace}

\title{\texttt{polymake} on the Mac\\\polymakeversion}

\newcommand{\polymake}{\texttt{polymake}\xspace}


\begin{document}

\maketitle
\parindent0pt

\section*{Introduction}

\polymake is a tool to study the combinatorics and the geometry of convex polytopes and polyhedra. It is also capable of dealing with simplicial complexes, matroids, polyhedral fans, graphs, tropical polytopes, and various other mathematical object.

\section*{System Requirements}

The \polymake package was built and tested on Intel-based Macs running Mac OS \macversion\ with Apple's command line tools using the \polymakeversion of \polymake. Due to system package dependencies it won't work on Macs with \textbf{any} other Mac OS X version or architecture.

Note that there are different packages for different perl versions. Make sure you download the version of the \polymake app that was built for the perl version that corresponds to your system perl. You can find the version number by executing \[\texttt{/usr/bin/perl --version}\] in a terminal.

If there is no version of the \polymake app matching your perl version please contact us at \href{http://forum.polymake.org}{\tt forum.polymake.org}.

\bigskip
The package requires that Apple's command line tools and java are installed on your Mac. If you don't have them a popup window will appear the first time polymake wants to use them. Please click install (you may need an administrator password to install them). The popoup for the command line tools is slightly confusing. It also offers to install XCode. This is not necessary. Just click \emph{Install}.

\section*{Installation}

Double clicking the \texttt{dmg} file mounts the disk image and opens it in the Finder. Drag \polymake to a suitable location on your system. Preferably this should be the standard \texttt{Applications} folder. You might want to save this README file somewhere. Afterwards you can eject the image and delete the  \texttt{dmg} file unless you want to rebuild the app.

To run \polymake double click the \polymake program icon. You can also drag the icon onto your dock for faster access. Note however that the \polymake bundle is not a true Mac application: it basically opens a \texttt{Terminal} and starts \polymake inside this. So it does not behave as most other apps, e.g. Mail: if you have a running \polymake session then clicking again on the Dock icon just opens another instance of \polymake instead of bringing the existing window to the front. Instead, you should navigate to the \texttt{Terminal} app (and then possibly cycle between different \texttt{Terminal} windows with \texttt{Cmd-<}) to reach \polymake).

\section*{Usage}

Double click the \polymake program icon to start \polymake. This will open a terminal and launch the interactive shell of \polymake inside the terminal. For an introduction to \polymake and the interactive shell see \href{http://polymake.org}{\tt polymake.org}\;.

To exit \polymake type \texttt{exit;} at the \polymake prompt (observe the ``;'' that is necessary for each \polymake command). Depending on your \texttt{Terminal} settings the \texttt{Terminal} window might not close but just tell you that the process is completed. In that case close the window manually (but wait until \polymake has finished, which might take a moment). You can change the behavior of \texttt{Terminal} in the \textit{Preferences} menu.  You cannot close \polymake by right clicking the Dock icon or via Expos\`e. \polymake saves open files and customization settings when you exit, so you \textbf{should not} terminate \polymake by closing the terminal.

\section*{Using \polymake with a jupyter notebook}

You can use polymake in a jupyter notebook. At the first start the bundle creates a configuration file \texttt{\$HOME/.polymake-macbundle/bundle.config} inside the general polymake customization folder (observe the dot!). In this file you can set the variable \texttt{POLYMAKE\_START\_IN\_JUPYTER} to 1. Note that running polymake inside a jupyter notebook requires a working installation of \texttt{\bf python3} and \texttt{\bf jupyter} that can be found via your path variable (or add it the path to \texttt{POLYMAKE\_ADD\_PATH} in the configuration file). These two programs are \textbf{not} part of the polymake bundle, you have to install them yourself, e.g. via \texttt{brew} or \texttt{fink}.

With the next start of the bundle polymake will start a jupyter notebook server in a terminal and open a notebook in your default web browser. You can now choose polymake in the menu \texttt{new} at the top right corner of the screen. At the first start this will ask your permission to install the polymake jupyter kernel in \texttt{\$HOME/.local} in your home directory and to launch a server. Subsequent calls will directly start the server. 

\textbf{Note}: You have to shut down the jupyter server yourself with \texttt{Ctrl-c} (twice) in the terminal started by polymake after you have exited the polymake notebook. 

There are several more variables set in this file to control how jupyter is started.
\begin{itemize}
\item \texttt{POLYMAKE\_JUPYTER\_NOTEBOOK\_DIR}: This sets the default directory for your jupyter notebook files. By default, this is your home directory. It is highly recommended to set this to some other directory. The given path can either be absolute or relative to your home directory (e.g. \texttt{$\sim$/Documents}). Note that the given directory must already exist, it will not be created for you.
\item \texttt{POLYMAKE\_JUPYTER\_IP}: The ip the jupyter server connects to. Usually this is \texttt{localhost}, but you can set this to something else if you start the server remotely. If you start it in a virtual maschine on your computer and want to connect to it from the host, then \texttt{0.0.0.0} is usually the right choice (note that in this case you must forward the jupyter port from the machine to the host).
\item \texttt{POLYMAKE\_JUPYTER\_PORT}: The port jupyter connects to. Usually \texttt{8888}.
\item \texttt{POLYMAKE\_JUPYTER\_START\_BROWSER}: Set this to 0 if jupyter should not start a session in a browser. You should choose this option if you start the server remotely or in a virtual machine. 
\item \texttt{POLYMAKE\_JUPYTER\_BROWSER}: Sets the default browser. You need to provide the full path to the binary, followed by \%s. See the configuration file for an example.
\end{itemize}

\section*{Customization Files}

\subsection*{\polymake customization files}

\polymake puts its customization files into the directory 

\[\texttt{.polymake-macbundle/} \]

in your home directory. The directory is created at the first start of \polymake. If you want to reset \polymake to default values then you can just delete this directory (You cannot do this in the Finder as it is a hidden folder (observe the ``.'' in front). Open a \texttt{Terminal}, \texttt{cd} into your home directory and type \texttt{rm -rf .polymake-macbundle}.).

These files, as well as the customization files written to your extension directory when you import an extension into polymake contain the absolute path to the directory containing the polymake files. Thus, after you have started \polymake once you cannot move the bundle to a different location on you Mac (e.g.\ from the disk image into your Application folder). You can, however, create links.

If you have to move the \polymake app bundle, then you have to delete the folder \texttt{.polymake-macbundle} in your home directory (see above for instructions), and do a \texttt{make distclean} in each extension base directory prior to calling \polymake from its new location. If you want to save some values (e.g.\ color settings) in \texttt{customize.pl} or \texttt{prefer.pl} then make a copy of the files and add the values back to the corresponding files in \texttt{.polymake-macbundle} after you have started \polymake from the new location.

\polymake has a basic check to detect whether you have moved the \polymake app and offers to delete \texttt{.polymake-macbundle}. You can accept this as long as you don't have imported any extensions. If you have, then please choose ``cancel'' here and do \texttt{make clean} in the extension base directories before you start \polymake\ from its new location.

\subsection*{Bundle specific customization}

At the first start the bundle adds its own customization file \texttt{bundle.config} to the directory {\$HOME/.polymake-macbundle}. This is mostly used to control the configuration of the jupyter server for using polymake in a jupyter notebook. 

There is one general option that lets you define the terminal app that should be used. By default, polymake starts inside the default \texttt{Terminal.app} that comes with MacOS. If you prefer to use \texttt{iTerm}, then you can set \texttt{POLYMAKE\_TERM='iTerm'} in that file. 


\section*{Uninstalling polymake}

To uninstall \polymake just drag the \polymake icon to the trash and remove the directory \texttt{\${HOME}/.polymake-macbundle} (This is only created after you have started \polymake at least once). Note that this directory is a hidden directory, so you cannot delete it in the Finder. Instead, open a \texttt{Terminal}, \texttt{cd} into your home directory, and type \texttt{rm -rf .polymake-macbundle} (observe the ``.'').

If you have installed the jupyter kernel for \polymake and want to remove this as well, then just delete the directory of the kernel. You can find this via
\[\texttt{jupyter kernelspec list}\]
in a terminal. The kernel is installed into a directory \texttt{Jupyter/kernels/polymake} below the jupyter kernel default directory.

\section*{Trouble-shooting}

\textbf{Please note that the \polymake app is not relocatable after the first start}, so please move it to a location on your hard disk (e.g.\ the Application folder) before starting it. If you need to move the app then please follow the instructions given in the section about customization files. You can use the \polymake app along with a standard installation, but you cannot import extensions from the same directory (but you can install extensions again to a different directory).

Support queries concerning installation and usage are welcome (please use our forum at \href{http://forum.polymake.org}{\tt forum.polymake.org}), as well as any other feedback, but are served on voluntary base, depending, first of all, on the authors' free time resources.  The \polymake package for Mac is still experimental, so it might not work on your computer. Also, we don't have many different Mac OS X installations at hand to test. If it doesn't work we'd value feedback about what went wrong. To obtain relevant information you could try to start the script from a terminal instead of the Applications folder and send us the output. If you have installed \polymake into the standard \texttt{Applications} folder, then the steps are as follows.
\begin{enumerate}
\item open the \texttt{Terminal} application (inside \texttt{Utilities} in your \texttt{Applications} folder).
\item at the prompt type\\
 \texttt{. /Application/polymake.app/Contents/MacOS/polymake.start}
\end{enumerate}

\section*{License}

\polymake is released under the the GPL license. By downloading \polymake in any form (whether source code or compiled) you agree to be bound by this license; further you renounce to claim any kind of warranty or damages related to the use of this software.

Software libraries bundled directly with \polymake are protected by open source licenses adequate to the GPL or broader. However, the exact wording and restrictions to use may vary. 

Additionally, the \polymake application package comes with compiled versions of several packages necessary for polymake: 
\begin{enumerate}
\item \href{GMPHOME}{\texttt{GMP GMPVERSION}}.
\item \href{MPFRHOME}{\texttt{MPFR MPFRVERSION}}.
\item \href{READLINEHOME}{\texttt{readline READLINEVERSION}}
\item \href{TERMRLGNUHOME}{\texttt{perl::Term-Readline-Gnu TERMRLGNUVERSION}}
\item \href{BOOSTHOME}{\texttt{boost BOOSTVERSION}}
\item \href{ANTHOME}{\texttt{ant ANTVERSION}}
\item \href{PPLHOME}{\texttt{ppl PPLVERSION}}
\item \href{NORMALIZHOME}{\texttt{libnormaliz NORMALIZVERSION}}
\item \href{CDDHOME}{\texttt{cdd CDDVERSION}}
\item \href{LRSHOME}{\texttt{lrs LRSVERSION}}
\item \href{NAUTYHOME}{\texttt{nauty NAUTYVERSION}}
\item \href{JREALITYHOME}{\texttt{jReality JREALITYVERSION}}
\item \href{LIBXSLTHOME}{\texttt{XML-LibXSLT LIBXSLTVERSION}}
\item \href{PERMLIBHOME}{\texttt{permlib PERMLIBVERSION}}
\item \href{NTLHOME}{\texttt{ntl NTLVERSION}}
\item \href{4TI2HOME}{\texttt{4ti2 4TI2VERSION}}
\item \href{GLPKHOME}{\texttt{glpk GLPKVERSION}}
\item \href{SINGULARHOME}{\texttt{Singular SINGULARVERSION}}
\item \href{NINJAHOME}{\texttt{ninja NINJAVERSION}}
\item \href{POLYMAKEHOME}{\texttt{polymake POLYMAKEVERSION}}
\end{enumerate}
Also these packages are protected by open source licences compliant to the GPL. The sources for this version of \polymake are also available from our download page at \href{http://polymake.org/doku.php/download/start}{polymake.org/doku.php/download/start}. All sources are either directly contained in the polymake distribution or included as compressed tar archives in the \texttt{tarballs/} directory of the disk image, where you also find the polymake sources. Check the corresponding COPYING or README files included in the packages for the exact license.  For the packages bundled directly with \polymake you can find the license statements in the \texttt{bundled} sub-directory of the \polymake archive (currently \texttt{cdd, lrs, nauty, permlib, ppl, libnormaliz, jreality}).

\section*{Rebuilding the disk-image}

If you need to rebuild the disk-image just copy the \texttt{src} directory to some location on your Mac. Change to the \texttt{src} directory and type \texttt{make fetch\_sources \&\& make compile \&\& make dmg}.

\end{document}
